\input hyperbasics
% standard macros for WEB listings (in addition to PLAIN.TEX)
% $Revision: 1.1 $ % Don Knuth, July 1990

\parskip 0pt % no stretch between paragraphs
\parindent 1em % for paragraphs and for the first line of C text

\font\ninerm=cmr9
\let\mc=\ninerm % medium caps
\def\Cee{{\mc C}}
\def\UNIX{{\mc UNIX}}
\font\eightrm=cmr8
\let\sc=\eightrm % small caps (NOT a caps-and-small-caps font)
\let\mainfont=\tenrm
\font\titlefont=cmr7 scaled\magstep4 % title on the contents page
\font\ttitlefont=cmtt10 scaled\magstep2 % typewriter type in title
\font\tentex=cmtex10 % TeX extended character set (used in strings)
\fontdimen7\tentex=0pt % no double space after sentences

\def\\#1{\leavevmode\hbox{\it#1\/\kern.05em}} % italic type for identifiers
\def\|#1{\leavevmode\hbox{$#1$}} % one-letter identifiers look better this way
\def\&#1{\leavevmode\hbox{\bf
  \def\_{\kern.04em\vbox{\hrule width.3em height .6pt}\kern.08em}%
  #1\/\kern.05em}} % boldface type for reserved words
\def\.#1{\leavevmode\hbox{\tentex % typewriter type for strings
  \let\\=\BS % backslash in a string
  \let\{=\LB % left brace in a string
  \let\}=\RB % right brace in a string
  \let\~=\TL % tilde in a string
  \let\ =\SP % space in a string
  \let\_=\UL % underline in a string
  \let\&=\AM % ampersand in a string
  \let\^=\CF % circumflex in a string
  #1}}
\def\){\discretionary{\hbox{\tentex\BS}}{}{}}
\def\AT{@} % at sign for control text

\chardef\AM=`\& % ampersand character in a string
\chardef\BS=`\\ % backslash in a string
\chardef\LB=`\{ % left brace in a string
\chardef\RB=`\} % right brace in a string
\def\SP{{\tt\char`\ }} % (visible) space in a string
\chardef\TL=`\~ % tilde in a string
\chardef\UL=`\_ % underline character in a string
\chardef\CF=`\^ % circumflex character in a string

\newbox\PPbox % symbol for ++
\setbox\PPbox=\hbox{\kern.5pt\raise1pt\hbox{\sevenrm+\kern-1pt+}\kern.5pt}
\def\PP{\copy\PPbox}
\newbox\MMbox \setbox\MMbox=\hbox{\kern.5pt\raise1pt\hbox{\sevensy\char0
 \kern-1pt\char0}\kern.5pt}
\def\MM{\copy\MMbox}
\newbox\MGbox % symbol for ->
\setbox\MGbox=\hbox{\kern-2pt\lower3pt\hbox{\teni\char'176}\kern1pt}
\def\MG{\copy\MGbox}
\let\GG=\gg
\let\LL=\ll
\let\NULL=\Lambda
\mathchardef\AND="2026 % bitwise and; also \& (unary operator)
\let\OR=\mid % bitwise or
\let\XOR=\oplus % bitwise exclusive or
\def\CM{{\sim}} % bitwise complement
\newbox\MODbox \setbox\MODbox=\hbox{\eightrm\%}
\def\MOD{\mathbin{\copy\MODbox}}

\newbox\bak \setbox\bak=\hbox to -1em{} % backspace one em
\newbox\bakk\setbox\bakk=\hbox to -2em{} % backspace two ems

\newcount\ind % current indentation in ems
\def\1{\global\advance\ind by1\hangindent\ind em} % indent one more notch
\def\2{\global\advance\ind by-1} % indent one less notch
\def\3#1{\hfil\penalty#10\hfilneg} % optional break within a statement
\def\4{\copy\bak} % backspace one notch
\def\5{\hfil\penalty-1\hfilneg\kern2.5em\copy\bakk\ignorespaces}% optional break
\def\6{\ifmmode\else\par % forced break
  \hangindent\ind em\noindent\kern\ind em\copy\bakk\ignorespaces\fi}
\def\7{\Y\6} % forced break and a little extra space
\def\8{\hskip-\ind em\hskip 2em} % no indentation

\let\yskip=\smallskip
\def\?{\mathrel?}
\let\hyperdoend=\relax
\def\note#1{\Y\noindent{\def\hyperdoend{.\par}%
            \hangindent2em\baselineskip10pt\eightrm#1%
            \catcode`\,=\active\let\hyperjunk=}~\secref%
            }
\def\lapstar{\rlap{*}}
\def\stsec{\Q\noindent{\let\*=\empty\bf\hyperdef\hypernoname
           {section}{\modstar}{\let\*=\lapstar\modstar.}\quad}}
\let\startsection=\stsec
\def\defin#1{\global\advance\ind by 2 \1\&{#1 }} % begin `define' or `format'
\def\A{\note{See also section}} % crossref for doubly defined section name
\def\As{\note{See also sections}} % crossref for multiply defined section name
%\def\B{\mathopen{\.{@\{}}} % begin controlled comment
\def\C#1{\5\5\quad$/\ast\,$#1$\,\ast/$}
\def\D{\defin{\#define}} % macro definition
\def\ET{ and~} % conjunction between two section numbers
\def\ETs{, and~} % conjunction between the last two of several section numbers
\def\F{\defin{format}} % format definition
\let\G=\ge % greater than or equal sign
\let\I=\ne % unequal sign
\def\J{\.{@\&}} % TANGLE's join operation
\let\K== % can be changed to left arrow, if desired
\let\L=\le % less than or equal sign
\outer\def\M#1.{\MN#1.\ifon\vfil\penalty-100\vfilneg % beginning of section
  \vskip12ptminus3pt\startsection\ignorespaces}
\outer\def\N#1.#2.{\MN#1.\vfil\eject % beginning of starred section
  \def\rhead{\uppercase{\ignorespaces#2}} % define running headline
  \message{*\modno} % progress report
  \edef\next{\write\cont{\Z{#2}{\modno}{\the\pageno}}}\next % to contents file
  \ifon\startsection{\bf\ignorespaces#2.\quad}\ignorespaces}
\def\MN#1.{\par % common code for \M, \N
  {\xdef\modstar{#1}\let\*=\empty\xdef\modno{#1}}
  \ifx\modno\modstar \onmaybe \else\ontrue \fi \mark{\modno}}
\def\O#1{\leavevmode % octal, hex or decimal constant
  \hbox{${\def\?{\kern.2em}%
  \def\${\ell}% long constant
  \def\_{\cdot 10^{\aftergroup}}% power of ten
  \def\~{\hbox{\rm\char'23\kern-.2em\it\aftergroup\?\aftergroup}}% octal
  \def\^{\hbox{\rm\char"7D\tt\aftergroup}}#1}$}}% double quotes for hex constant
\def\P{\rightskip=0pt plus 100pt minus 10pt % go into C mode
  \sfcode`;=3000
  \pretolerance 10000
  \hyphenpenalty 9999 % so strings can be broken (a discretionary \ is inserted)
  \exhyphenpenalty 10000
  \global\ind=2 \1\ \unskip}
\def\Q{\rightskip=0pt % get out of C mode
  \sfcode`;=1500 \pretolerance 200 \hyphenpenalty 50 \exhyphenpenalty 50 }
\let\R=\lnot % logical not
\let\S=\equiv % equivalence sign
%\def\T{\mathclose{\.{@\ast/}}} % terminate controlled comment
\def\U{\note{This code is used in section}} % xref for use of a section
\def\Us{\note{This code is used in sections}} % xref for uses of a section
\let\V=\lor % logical or
\let\W=\land % logical and
\def\X{{\iffalse}\fi\catcode`\,=\active\relax\Xone}
\def\Xone#1:{\iffalse{\fi}\Xtwo#1:}
\def\Xtwo#1:#2\X{\ifmmode\gdef\XX{\null$\null}\else\gdef\XX{}\fi % section name
  \XX$\langle\,${#2\eightrm\kern.5em{\def\hyperdoend{}%
  \secref#1.}\iffalse}\fi$\,\rangle$\XX}
\def\Y{\par\yskip}
\let\Z=\let % now you can \send the control sequence \Z
\def\=#1{\leavevmode\hbox{\kern2pt\vrule\vtop{\vbox{\hrule
        \hbox{\strut\kern2pt\.{#1}\kern2pt}}
      \hrule}\vrule\kern2pt}} % verbatim string
\let\*=*

\def\onmaybe{\let\ifon=\maybe} \let\maybe=\iftrue
\newif\ifon \newif\iftitle \newif\ifpagesaved
\def\lheader{\mainfont\hyperdef\hypernoname{page}{\the\pageno}%
             {\the\pageno}\eightrm\qquad\rhead\hfill\title\qquad
  \tensy x\mainfont\topmark} % top line on left-hand pages
\def\rheader{\tensy x\mainfont\topmark\eightrm\qquad\title\hfill\rhead
  \qquad\mainfont\hyperdef\hypernoname{page}{\the\pageno}%
             {\the\pageno}} % top line on right-hand pages
\def\page{\box255 }
\newbox\hyperbox
\newif\iffooter\footerfalse
\def\normaloutput#1#2#3{\shipout\vbox{
  \ifodd\pageno\hoffset=\pageshift\fi
  \vbox to\fullpageheight{
  \iftitle\global\titlefalse
  \else\hbox to\pagewidth{\vbox to10pt{}\ifodd\pageno #3\else#2\fi}\fi
  \vfill#1\iffooter\vfill\box\hyperbox\fi}} % parameter #1 is the page itself
  \global\advance\pageno by1}

\def\rhead{\.{CWEB} OUTPUT} % this running head is reset by starred sections
\def\title{} % an optional title can be set by the user
\def\topofcontents{\centerline{\titlefont\title}
  \vfill} % this material will start the table of contents page
\def\botofcontents{\vfill} % this material will end the table of contents page
\def\contentspagenumber{0} % default page number for table of contents
\newdimen\pagewidth \pagewidth=6.5in % the width of each page
{\boxmaxdepth=0pt\relax\global
\setbox\hyperbox\hbox to \pagewidth{GO TO:\hfil\hyperref{}{page}{1}{first 
      page}\hfil\hyperref{}{section}{INDEX}{Index}\hfil\hyperref{}{section}%
      {SECTIONS}{Section Names}\hfil\hyperref{}{section}{CONTENTS}{Contents}}}
\newdimen\pageheight \pageheight=8.7in % the height of each page
\iffooter\advance\pageheight by -\ht\hyperbox\relax\fi
\newdimen\fullpageheight \fullpageheight=9in % page height including headlines
\newdimen\pageshift \pageshift=0in % shift righthand pages wrt lefthand ones
\def\magnify#1{\mag=#1\pagewidth=6.5truein\pageheight=8.7truein
  \fullpageheight=9truein\setpage}
\def\setpage{\hsize\pagewidth\vsize\pageheight} % use after changing page size
\def\contentsfile{\jobname.toc} % file that gets table of contents info
\def\readcontents{\input \jobname.toc}

\newwrite\cont
\output{\setbox0=\page % the first page is garbage
  \openout\cont=\contentsfile
       \write\cont{\catcode `\noexpand\@=11\relax}   % \makeatletter
  \global\output{\normaloutput\page\lheader\rheader}}
\setpage
\vbox to \vsize{} % the first \topmark won't be null

\def\ch{\note{The following sections were changed by the change file:}
  \let\*=\relax}
\newbox\sbox % saved box preceding the index
\newbox\lbox % lefthand column in the index
\def\comma{,}
\def\hypernoblanks#1{#1}
\def\hypereatup#1{}
{\global\futurelet\hyperfunny} %
\def\hyperglobal{\ifx\hyperfunny\hyperblank
                 \global\futurelet\hyperblank\hypereatup\hyperblank
                 \else\global\let\hyperblank=\relax\fi}
\def\hyperblankonly#1{{\setbox0=\hbox{\futurelet\hyperblank\hyperglobal#1}%
                      \hyperblank}}
\def\hypersecref#1\hyperjunk{\def\[##1]{##1}\hyperblankonly{#1}%
               \let\preserve=\*\let\*=\empty
               \hyperref{}{section}%
               {\hypernoblanks#1}{\let\*=\preserve\let\[=\hyperrestore\hypernoblanks#1}}%
{\catcode`,=\active\relax
\gdef\secref#1.{\let\hyperjunk=\relax%
               \edef,{\hyperjunk\comma\noexpand\noexpand\noexpand\hypersecref}%
               \edef\ET{\hyperjunk\ET\noexpand\noexpand\noexpand\hypersecref}%
               \edef\ETs{\hyperjunk\ETs\noexpand\noexpand\noexpand\hypersecref}%
               \let\hyperrestore=\[\let\[=\relax%]]
               {\let\hyperblankonly=\relax
               \xdef\expanded{#1\hyperjunk}}%
               \expandafter\hypersecref\expanded\let\hyperjunk={\hyperdoend}%
               }%
}
\def\inx{\par\vskip6pt plus 1fil % we are beginning the index
  \write\cont{} % ensure that the contents file isn't empty
       \write\cont{\catcode `\noexpand\@=12\relax}   % \makeatother
  \closeout\cont % the contents information has been fully gathered
  \output{\ifpagesaved\normaloutput{\box\sbox}\lheader\rheader\fi
    \global\setbox\sbox=\page \global\pagesavedtrue}
  \pagesavedfalse \eject % eject the page-so-far and predecessors
  \setbox\sbox\vbox{\unvbox\sbox} % take it out of its box
  \vsize=\pageheight \advance\vsize by -\ht\sbox % the remaining height
  \hsize=.5\pagewidth \advance\hsize by -10pt
    % column width for the index (20pt between cols)
  \parfillskip 0pt plus .6\hsize % try to avoid almost empty lines
  \def\lr{L} % this tells whether the left or right column is next
  \output{\if L\lr\global\setbox\lbox=\page \gdef\lr{R}
    \else\normaloutput{\vbox to\pageheight{\box\sbox\vss
        \hbox to\pagewidth{\box\lbox\hfil\page}}}\lheader\rheader
    \global\vsize\pageheight\gdef\lr{L}\global\pagesavedfalse\fi}
  \message{Index:}
  \parskip 0pt plus .5pt
  \outer\def\:##1, {\par\hangindent2em\noindent##1:{\kern1em
  \catcode`\,=\active\relax\iffalse}\fi\def\hyperdoend{}\secref} % index entry
  \def\[##1]{$\underline{##1}$} % underlined index item
  \hyperdef\hypernoname{section}{INDEX}{}%
  \rm \rightskip0pt plus 2.5em \tolerance 10000 \let\*=\lapstar
  \hyphenpenalty 10000 \parindent0pt}
\def\fin{\par\vfill\eject % this is done when we are ending the index
  \ifpagesaved\null\vfill\eject\fi % output a null index column
  \if L\lr\else\null\vfill\eject\fi % finish the current page
  \parfillskip 0pt plus 1fil
  \def\rhead{NAMES OF THE SECTIONS}
  \message{Section names:}
  \output{\normaloutput\page\lheader\rheader}
  \setpage
  \def\note##1{\quad{\eightrm##1~%
               \catcode`\,=\active\let\hyperjunk=}\def\hyperdoend{}\secref}
  \hyperdef\hypernoname{section}{SECTIONS}{}%
  \def\U{\note{Used in section}} % crossref for use of a section
  \def\Us{\note{Used in sections}} % crossref for uses of a section
  \def\:{\par\hangindent 2em}\let\*=*}
\def\con{\par\vfill\eject % finish the section names
  \rightskip 0pt \hyphenpenalty 50 \tolerance 200
  \setpage
  \output{\normaloutput\page\lheader\rheader}
  \titletrue % prepare to output the table of contents
  \pageno=\contentspagenumber \def\rhead{TABLE OF CONTENTS}
  \message{Table of contents:}
  \hyperdef\hypernoname{section}{CONTENTS}{}
  \topofcontents
  \line{\hfil Section\hbox to3em{\hss Page}}
  \def\Z##1##2##3{\line{\ignorespaces##1
    \leaders\hbox to .5em{.\hfil}\hfil\ \let\preserve=\*\let\*=\empty
    \hyperref{}{section}{##2}{\let\*=\preserve##2}%
    \hbox to3em{\hss\hyperref{}{page}{##3}{##3}}}}
  \readcontents\relax % read the contents info
  \botofcontents \end} % print the contents page(s) and terminate
\def\noinx{\let\inx=\end} % no indexes or table of contents
\def\nomods{\let\FIN=\fin \def\fin{\let\parfillskip=\end \FIN}}
    % no index of module names or table of contents
\def\nocon{\let\con=\end} % no table of contents
\def\today{\ifcase\month\or
  January\or February\or March\or April\or May\or June\or
  July\or August\or September\or October\or November\or December\fi
  \space\number\day, \number\year}
\newcount\twodigits
\def\hours{\twodigits=\time \divide\twodigits by 60 \printtwodigits
  \multiply\twodigits by-60 \advance\twodigits by\time \printtwodigits}
\def\gobbleone1{}
\def\printtwodigits{\advance\twodigits100
  \expandafter\gobbleone\number\twodigits
  \advance\twodigits-100 }
\def\datethis{\def\startsection{\leftline{\sc\today\ at \hours}\bigskip
  \let\startsection=\stsec\stsec}}
  % say `\datethis' in limbo, to get your listing timestamped
